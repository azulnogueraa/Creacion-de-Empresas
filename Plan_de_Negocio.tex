% Options for packages loaded elsewhere
\PassOptionsToPackage{unicode}{hyperref}
\PassOptionsToPackage{hyphens}{url}
%
\documentclass[
]{article}
\usepackage{amsmath,amssymb}
\usepackage{iftex}
\ifPDFTeX
  \usepackage[T1]{fontenc}
  \usepackage[utf8]{inputenc}
  \usepackage{textcomp} % provide euro and other symbols
\else % if luatex or xetex
  \usepackage{unicode-math} % this also loads fontspec
  \defaultfontfeatures{Scale=MatchLowercase}
  \defaultfontfeatures[\rmfamily]{Ligatures=TeX,Scale=1}
\fi
\usepackage{lmodern}
\ifPDFTeX\else
  % xetex/luatex font selection
\fi
% Use upquote if available, for straight quotes in verbatim environments
\IfFileExists{upquote.sty}{\usepackage{upquote}}{}
\IfFileExists{microtype.sty}{% use microtype if available
  \usepackage[]{microtype}
  \UseMicrotypeSet[protrusion]{basicmath} % disable protrusion for tt fonts
}{}
\makeatletter
\@ifundefined{KOMAClassName}{% if non-KOMA class
  \IfFileExists{parskip.sty}{%
    \usepackage{parskip}
  }{% else
    \setlength{\parindent}{0pt}
    \setlength{\parskip}{6pt plus 2pt minus 1pt}}
}{% if KOMA class
  \KOMAoptions{parskip=half}}
\makeatother
\usepackage{xcolor}
\usepackage[margin=1in]{geometry}
\usepackage{longtable,booktabs,array}
\usepackage{calc} % for calculating minipage widths
% Correct order of tables after \paragraph or \subparagraph
\usepackage{etoolbox}
\makeatletter
\patchcmd\longtable{\par}{\if@noskipsec\mbox{}\fi\par}{}{}
\makeatother
% Allow footnotes in longtable head/foot
\IfFileExists{footnotehyper.sty}{\usepackage{footnotehyper}}{\usepackage{footnote}}
\makesavenoteenv{longtable}
\usepackage{graphicx}
\makeatletter
\def\maxwidth{\ifdim\Gin@nat@width>\linewidth\linewidth\else\Gin@nat@width\fi}
\def\maxheight{\ifdim\Gin@nat@height>\textheight\textheight\else\Gin@nat@height\fi}
\makeatother
% Scale images if necessary, so that they will not overflow the page
% margins by default, and it is still possible to overwrite the defaults
% using explicit options in \includegraphics[width, height, ...]{}
\setkeys{Gin}{width=\maxwidth,height=\maxheight,keepaspectratio}
% Set default figure placement to htbp
\makeatletter
\def\fps@figure{htbp}
\makeatother
\setlength{\emergencystretch}{3em} % prevent overfull lines
\providecommand{\tightlist}{%
  \setlength{\itemsep}{0pt}\setlength{\parskip}{0pt}}
\setcounter{secnumdepth}{-\maxdimen} % remove section numbering
\ifLuaTeX
\usepackage[bidi=basic]{babel}
\else
\usepackage[bidi=default]{babel}
\fi
\babelprovide[main,import]{spanish}
% get rid of language-specific shorthands (see #6817):
\let\LanguageShortHands\languageshorthands
\def\languageshorthands#1{}
\ifLuaTeX
  \usepackage{selnolig}  % disable illegal ligatures
\fi
\usepackage{bookmark}
\IfFileExists{xurl.sty}{\usepackage{xurl}}{} % add URL line breaks if available
\urlstyle{same}
\hypersetup{
  pdftitle={Plan de Negocio},
  pdfauthor={Azul Noguera; Patricio Guledjian; Rocio Gonzalez Cingolani; Rafael Cabre},
  pdflang={es-ES},
  hidelinks,
  pdfcreator={LaTeX via pandoc}}

\title{Plan de Negocio}
\usepackage{etoolbox}
\makeatletter
\providecommand{\subtitle}[1]{% add subtitle to \maketitle
  \apptocmd{\@title}{\par {\large #1 \par}}{}{}
}
\makeatother
\subtitle{Link al repositorio:
\url{https://github.com/azulnogueraa/Creacion-de-Empresas}}
\author{Azul Noguera \and Patricio Guledjian \and Rocio Gonzalez
Cingolani \and Rafael Cabre}
\date{2024-04-27}

\begin{document}
\maketitle

{
\setcounter{tocdepth}{3}
\tableofcontents
}
\newpage

\section*{Plan de Negocio}

\subsection{1. Introducción}\label{introducciuxf3n}

\subsubsection{1.1. CNAE (Clasificación Nacional de Entidades
Económicas)}\label{cnae-clasificaciuxf3n-nacional-de-entidades-econuxf3micas}

Educación.

\subsubsection{1.2. Misión}\label{misiuxf3n}

Ofrecer una educación personalizada y de alta calidad a través de la
tecnología, para que las personas puedan aprender de manera eficiente y
efectiva, mejorando la satisfacción de los estudiantes y las tasas de
finalización de los cursos.

\subsubsection{1.3. Visión}\label{visiuxf3n}

Ser la plataforma de educación en línea más grande y reconocida en el
mundo, ofreciendo una amplia variedad de cursos y programas adaptados a
las necesidades de cada estudiante.

\subsubsection{1.4. Valores}\label{valores}

\begin{itemize}
\tightlist
\item
  Calidad
\item
  Innovación
\item
  Personalización
\item
  Eficiencia
\item
  Ética
\end{itemize}

\newpage

\subsection{2. Descripción de la Oportunidad de
Negocio}\label{descripciuxf3n-de-la-oportunidad-de-negocio}

El avance tecnológico ha registrado una trayectoria ascendente sin
precedentes en las últimas décadas. Este progreso ha catalizado una
transformación paralela en el dominio educativo, propulsando el
e-learning a un primer plano en la entrega de conocimiento y
habilidades. La digitalización de la educación no solo ha democratizado
el acceso a la información, sino que también ha ampliado las fronteras
del aula tradicional, permitiendo un aprendizaje más flexible y
autodirigido.

En el vértice de la crisis sanitaria global, la pandemia de COVID-19 se
desempeño como catalizador para una adopción masiva de la educación en
línea. El confinamiento y la interrupción de la enseñanza presencial
urgieron a una migración acelerada hacia plataformas digitales. Según
una encuesta realizada en julio del año 2020 por la escuela de negocios
Online Business School (OBS)\footnote{FUENTE:
  \href{https://marketing.onlinebschool.es/Prensa/Informes/Informe\%20OBS\%20E-learning\%202022.pdf}{OBS
  Business School -El Estudiante Universitario en Línea. Tendencias y
  Perspectivas.}} de España, el 84\% de los 30 países analizados en la
región había elegido internet como uno de sus sistemas de educación a
distancia.

\begin{figure}
\centering
\includegraphics[width=0.7\textwidth,height=\textheight]{img/internet_ante_papel.png}
\caption{Sistemas de entrega de educación a distancia utilizados en
medio de la pandemia de COVID-19}
\end{figure}

Posteriormente, una vez atenuado el impacto directo de la pandemia, la
inercia del aprendizaje en línea perduró. Muchos estudiantes, ahora
acostumbrados al entorno digital, continuaron favoreciendo esta
modalidad, apoyados por la escalada tecnológica que seguía enriqueciendo
esta experiencia. El impulso continuo del e-learning responde también a
una demanda creciente de capacitación constante, vital en un mercado
laboral que se reinventa continuamente. Según estimaciones de Global
Market Insights, la valuación del mercado de e-learning en \$399.3 mil
millones de dolares en 2022 y su proyectado crecimiento a una tasa
compuesta anual del 14\% hasta 2032, subraya esta tendencia.

Sin embargo, el escenario actual de la educación en línea destaca un
marcado contraste entre la disponibilidad de recursos educativos y la
efectividad de su implementación. La mayoría de las plataformas
digitales persisten en ofrecer programas estandarizados, con falta de la
flexibilidad necesaria para atender las demandas personalizadas de
aprendizaje. Esta homogeneización del contenido pedagógico conlleva a
deficiencias en la retención de conocimientos por parte de los
estudiantes, evidenciado por las tasas de finalización de cursos que, de
manera preocupante, suelen rondar el 10\%\footnote{FUENTE:
  \href{https://fastercapital.com/es/contenido/Tasas-de-retencion-de-cursos--Tasas-de-retencion-de-cursos--una-metrica-clave-para-el-exito-del-marketing.html}{fastercapital.com}}.
Tal fenómeno no solo sugiere una desconexión entre la instrucción y la
diversidad de estilos de aprendizaje, sino que también plantea el riesgo
de desaprovechar el potencial académico de estudiantes que requerirían
una metodología más ajustada a sus particularidades. Este desafío en el
panorama educativo actual subraya la necesidad de una plataforma de
aprendizaje electrónico que privilegie la adaptabilidad y
personalización en su oferta curricular, alineándose estrechamente con
las aptitudes y aspiraciones individuales de cada estudiante.

No obstante, investigaciones recientes del año 2023 realizadas por la
Online Business School (OBS)\footnote{FUENTE:\href{https://www.obsbusiness.school/sites/obsbusiness.school/files/media_files/Informe\%20OBS\%20E-Learning\%202023.pdf}{OBS
  Business School - Tendencias y Percepciones sobre la Educación en
  Línea y la Adopción de Tecnologías Educativas}} de España anticipan un
aumento significativo en las inscripciones a programas en línea,
destacando el ámbito de la tecnología con un crecimiento proyectado del
33\% en los próximos cinco años. Esta tendencia subraya una oportunidad
primordial para desarrollar cursos en línea especializados en
tecnología. A continuación, se incluye una figura que ilustra
detalladamente estos hallazgos, fundamentando la viabilidad y el
potencial de esta iniciativa.

\begin{figure}
\centering
\includegraphics[width=0.7\textwidth,height=\textheight]{img/ingenieria_tecnologia.png}
\caption{Campos de estudio en los que se anticipa mayor crecimiento en
los programas en línea}
\end{figure}

En conclusión, la rápida evolución tecnológica y la transformación del
paisaje educativo presentan una oportunidad singular para innovar en la
entrega de educación en línea. La persistencia de desafíos, como la
estandarización excesiva y la baja tasa de finalización de cursos, no
solo destaca las deficiencias en las plataformas actuales sino que
también subraya el potencial para una plataforma que ofrezca soluciones
personalizadas y adaptativas. Aprovechando las tendencias emergentes y
las necesidades cambiantes del mercado laboral, se revela un campo
fértil para el desarrollo de una oferta educativa que no solo mejore la
experiencia del aprendizaje digital sino que también incremente la
efectividad del mismo. Con el respaldo de datos y proyecciones que
auguran un crecimiento robusto en la demanda de educación técnica en
línea, nuestro proyecto está posicionado estratégicamente para liderar
esta próxima ola de innovación educativa, asegurando que la educación en
línea sea más accesible, relevante y fructífera para todos los
estudiantes.

\newpage

\subsection{3. Entrevista Problema}\label{entrevista-problema}

\begin{itemize}
\tightlist
\item
  ¿Cuál es tu edad?

  \begin{enumerate}
  \def\labelenumi{\alph{enumi}.}
  \tightlist
  \item
    Menos de 17 años
  \item
    Entre 17 y 25 años
  \item
    Mayor a 25 años
  \end{enumerate}
\end{itemize}

\includegraphics[width=0.6\textwidth,height=\textheight]{img/años.png}

\vspace{10mm}

\begin{itemize}
\tightlist
\item
  ¿Estás actualmente estudiando?

  \begin{enumerate}
  \def\labelenumi{\alph{enumi}.}
  \tightlist
  \item
    Si
  \item
    No
  \end{enumerate}
\end{itemize}

\includegraphics[width=0.6\textwidth,height=\textheight]{img/estudiando.png}

\newpage

\begin{itemize}
\tightlist
\item
  ¿Cuál es tu nivel educativo actual?

  \begin{enumerate}
  \def\labelenumi{\alph{enumi}.}
  \tightlist
  \item
    Primaria
  \item
    Secundaria
  \item
    Universitaria
  \item
    Postgrado
  \item
    otra
  \end{enumerate}
\end{itemize}

\includegraphics[width=0.6\textwidth,height=\textheight]{img/nivel.png}

\vspace{10mm}

\begin{itemize}
\tightlist
\item
  ¿Qué dificultades has enfrentado con los métodos de enseñanza que
  frecuentas?

  \begin{enumerate}
  \def\labelenumi{\alph{enumi}.}
  \tightlist
  \item
    Falta de personalización
  \item
    Ritmo de aprendizaje inadecuado
  \item
    Dificultad para entender el contenido
  \item
    No he enfrentado dificultades
  \item
    Otra
  \end{enumerate}
\end{itemize}

\includegraphics[width=0.7\textwidth,height=\textheight]{img/dificultades.png}

\newpage

\begin{itemize}
\tightlist
\item
  ¿Consideras que tus necesidades individuales de aprendizaje están
  siendo satisfechas?

  \begin{enumerate}
  \def\labelenumi{\alph{enumi}.}
  \tightlist
  \item
    Si
  \item
    No
  \end{enumerate}
\end{itemize}

\includegraphics[width=0.5\textwidth,height=\textheight]{img/necesidades.png}

\vspace{10mm}

\begin{itemize}
\tightlist
\item
  ¿Que opinas de la IA?

  \begin{enumerate}
  \def\labelenumi{\alph{enumi}.}
  \tightlist
  \item
    No me gusta, prefiero hacer las cosas a mi manera.
  \item
    Lo uso de vez en cuando.
  \item
    Me parece estupenda la IA, y aprovecho mucho su capacidad.
  \end{enumerate}
\end{itemize}

\includegraphics[width=0.6\textwidth,height=\textheight]{img/IA.png}
\newpage

\begin{itemize}
\tightlist
\item
  ¿Has utilizado plataformas de educación en línea?

  \begin{enumerate}
  \def\labelenumi{\alph{enumi}.}
  \tightlist
  \item
    Si
  \item
    No
  \end{enumerate}
\end{itemize}

\includegraphics[width=0.6\textwidth,height=\textheight]{img/utilizado.png}

\vspace{10mm}

\begin{itemize}
\tightlist
\item
  En general, ¿terminas tus cursos?

  \begin{enumerate}
  \def\labelenumi{\alph{enumi}.}
  \tightlist
  \item
    Si, termino todos los cursos que comienzo.
  \item
    Varios me quedaron incompletos.
  \item
    Nunca logro completarlos.
  \end{enumerate}
\end{itemize}

\includegraphics[width=0.6\textwidth,height=\textheight]{img/terminar.png}

\newpage

\begin{itemize}
\tightlist
\item
  Si tu respuesta no fue la Opcion 1, ¿por que crees que no lograste
  terminar determinados cursos?
\end{itemize}

\includegraphics[width=0.7\textwidth,height=\textheight]{img/porque.png}

\vspace{10mm}

\begin{itemize}
\item
  Ahora responde teniendo en cuenta lo siguiente: Nosotros nos
  especializamos en ofrecer una plataforma de educación en línea
  altamente sofisticada, diseñada para brindar un aprendizaje
  personalizado y de calidad superior. Nuestra plataforma se distingue
  por su enfoque en la adaptabilidad, permitiendo que cada usuario
  reciba una experiencia educativa adaptada a sus preferencias y ritmos
  individuales de aprendizaje. Desde la selección de contenido hasta la
  interacción con instructores, nuestro objetivo es garantizar que cada
  estudiante pueda alcanzar su máximo potencial de manera efectiva y
  satisfactoria.
\item
  Sabiendo que nuestra plataforma se encuentra preparada con IA para
  resolver esos problemas, ¿que tan probable seria que te subscribas
  mensualmente?

  \begin{enumerate}
  \def\labelenumi{\alph{enumi}.}
  \tightlist
  \item
    1
  \item
    2
  \item
    3
  \item
    4
  \item
    5
  \end{enumerate}
\end{itemize}

\includegraphics[width=0.7\textwidth,height=\textheight]{img/ratting.png}

\newpage

\vspace{10mm}

\begin{itemize}
\tightlist
\item
  ¿Que caracteristicas buscas en una plataforma de educacion en linea?
\end{itemize}

\includegraphics[width=0.7\textwidth,height=\textheight]{img/caracteristicas.png}

\newpage

\begin{itemize}
\tightlist
\item
  ¿Como te gustaria interactuar con la IA en nuestra plataforma?

  \begin{enumerate}
  \def\labelenumi{\alph{enumi}.}
  \tightlist
  \item
    Recomendaciones personalizadas
  \item
    Asistentes virtual
  \item
    Adaptacion de ritmo y dificultad
  \item
    Todas las anteriores
  \end{enumerate}
\end{itemize}

\includegraphics[width=0.6\textwidth,height=\textheight]{img/interacción_IA.png}

\begin{itemize}
\tightlist
\item
  ¿Cuánto estarías dispuesto/a a pagar mensualmente en tu educación en
  línea?

  \begin{enumerate}
  \def\labelenumi{\alph{enumi}.}
  \tightlist
  \item
    De 1€ a 15€
  \item
    De 15€ a 50€
  \item
    De 50€ a 100€
  \item
    Más de 100€
  \end{enumerate}
\end{itemize}

\includegraphics[width=0.6\textwidth,height=\textheight]{img/precio.png}
\newpage

\begin{itemize}
\tightlist
\item
  ¿Recomendarias la aplicacion?

  \begin{enumerate}
  \def\labelenumi{\alph{enumi}.}
  \tightlist
  \item
    Si
  \item
    No
  \end{enumerate}
\end{itemize}

\includegraphics[width=0.6\textwidth,height=\textheight]{img/recomendación.png}

\newpage

\subsection{4. Mapa de Empatia}\label{mapa-de-empatia}

\subsubsection{4.1. ¿Qué ve?}\label{quuxe9-ve}

Ve desafíos académicos en sus cursos estándar, diferentes estilos de
aprendizaje y la necesidad de obtener buenos resultados académicos.

\subsubsection{4.2. ¿Qué escucha?}\label{quuxe9-escucha}

Escucha la frustración de otros estudiantes por la falta de
personalización en la enseñanza, a profesores sobre el contenido
estándar del curso.

\subsubsection{4.3. ¿Qué piensa y siente?}\label{quuxe9-piensa-y-siente}

Piensa en aprender de manera más efectiva, siente frustración cuando no
puede seguir el ritmo del curso y se preocupa por su desempeño académico
y su futuro profesional.

\subsubsection{4.4. ¿Qué dice y hace?}\label{quuxe9-dice-y-hace}

Expresa su deseo de cursos más adaptados, busca recursos adicionales
para aprender y participa en grupos de estudio o busca ayuda de
compañeros de clase.

\subsubsection{4.5. ¿Cuáles son sus problemas y
necesidades?}\label{cuuxe1les-son-sus-problemas-y-necesidades}

Necesita una educación que se adapte a su estilo de aprendizaje y ritmo,
obtener buenos resultados académicos y herramientas para comprender
mejor los temas difíciles.

\subsubsection{4.6. ¿Qué le motiva?}\label{quuxe9-le-motiva}

La motivación para obtener un título universitario y tener éxito
profesional, aprender nuevas habilidades y superar los desafíos
académicos.

\newpage

\subsection{5. Lienzo de Propuesta de
Valor}\label{lienzo-de-propuesta-de-valor}

\subsubsection{5.1. Tareas del cliente}\label{tareas-del-cliente}

Los clientes buscan obtener información precisa y actualizada, adquirir
habilidades prácticas aplicables y aprender de manera flexible. Esto
implica buscar y seleccionar cursos, inscribirse en ellos, gestionar su
progreso, interactuar con el contenido, participar en evaluaciones,
comunicarse con profesores, colaborar con otros estudiantes y establecer
conexiones. Además, desean mantenerse motivados, desarrollar confianza
en su capacidad y experimentar un sentido de logro al completar cursos y
alcanzar metas personales.

\subsubsection{5.2. Penas del Cliente}\label{penas-del-cliente}

Los clientes pueden enfrentar diversas dificultades al utilizar otra
plataforma, como la frustración por la navegación complicada o la
dificultad para encontrar información específica. También pueden
experimentar preocupación por la sobrecarga de información, temores
relacionados con costos ocultos y la insatisfacción con la calidad del
contenido.

\subsubsection{5.3. Alegrías del Cliente}\label{alegruxedas-del-cliente}

Nuestra plataforma ofrece varias ventajas y alegrías para los clientes.
Estos incluyen el ahorro de tiempo y dinero, y la comodidad. También, la
satisfacción de encontrar contenido relacionado con los intereses y
necesidades del cliente. Además, encontrarán gratificación personal al
alcanzar metas educativas y profesionales.

\subsubsection{5.4. Bienes y Servicios}\label{bienes-y-servicios}

Ofrecemos una amplia gama de servicios y recursos para satisfacer las
necesidades educativas de nuestros clientes. Esto incluye una variedad
de cursos en línea en el área de la tecnología, impartidos por expertos
en la materia, una plataforma intuitiva y fácil de usar con herramientas
interactivas para facilitar el aprendizaje, y un sólido soporte técnico
y académico. Además, nos comprometemos a mantener el contenido
actualizado y a mejorar continuamente nuestra oferta educativa.

\subsubsection{5.5. Creadores de
Alegrías}\label{creadores-de-alegruxedas}

Nuestro equipo se esfuerza por crear una experiencia educativa positiva
y enriquecedora para nuestros clientes. Esto implica ofrecer
flexibilidad en términos de horarios y ubicación, así como la
posibilidad de personalizar el aprendizaje según las preferencias
individuales. También nos comprometemos a proporcionar una amplia
variedad y calidad de contenido educativo, una interfaz intuitiva y
fácil de usar, y una comunidad de apoyo activa para enriquecer la
experiencia de aprendizaje.

\subsubsection{5.6. Quitapenas}\label{quitapenas}

Nos esforzamos por abordar las preocupaciones y dificultades que pueden
surgir al utilizar nuestra plataforma. Esto incluye mejorar la interfaz
y la experiencia del usuario para hacerla más intuitiva y fácil de usar,
así como garantizar la transparencia en los costos y ofrecer un soporte
ampliado. Además, valoramos el feedback de nuestros usuarios y nos
comprometemos a realizar mejoras continuas para satisfacer sus
necesidades y expectativas en términos de calidad y relevancia
educativa.

\newpage

\subsection{6. Modelo de Negocio}\label{modelo-de-negocio}

\subsubsection{6.1. Socios clave}\label{socios-clave}

\begin{itemize}
\tightlist
\item
  Educadores y universidades que proveen cursos y material didáctico.
\item
  Especialistas en tecnología que respaldan la infraestructura de la
  plataforma, incluyendo inteligencia artificial para personalización
  del aprendizaje.
\item
  Organizaciones de Certificación que validan y dan prestigio a los
  certificados ofrecidos por los cursos completados en la plataforma.
\item
  Fuentes financieras que apoyan la sostenibilidad y expansión de la
  plataforma.
\end{itemize}

\subsubsection{6.2. Actividades clave}\label{actividades-clave}

\begin{itemize}
\tightlist
\item
  Actualización de Contenidos: Evaluación y selección de materiales
  educativos que se alineen con las necesidades de los usuarios.
\item
  Marketing y Promoción: Estrategias de marketing digital para atraer a
  nuevos usuarios y retener a los existentes.
\item
  Soporte y Servicio al Cliente: Asistencia continua a usuarios para
  resolver problemas técnicos o dudas académicas.
\end{itemize}

\subsubsection{6.3. Recursos Clave}\label{recursos-clave}

\paragraph{6.3.1. Tangibles}\label{tangibles}

\begin{itemize}
\tightlist
\item
  Físicos: Infraestructura de hardware necesaria para soportar y alojar
  la plataforma de educación en línea, como servidores potentes, equipos
  de cómputo y sistemas de almacenamiento de datos.
\item
  Económicos-financieros: Capital necesario para mantener la plataforma,
  promocionar el servicio educativo y pagar al equipo humano.
\end{itemize}

\paragraph{6.3.2 Intangibles}\label{intangibles}

\begin{itemize}
\tightlist
\item
  Equipo Humano: Un equipo de profesionales que incluye desarrolladores
  web, creadores de contenido, especialistas en marketing, personal de
  soporte y personal educativo.
\item
  Asistencia personal: Ofrecer asistencia individualizada para resolver
  dudas o problemas técnicos.
\item
  Comunidad de Aprendizaje: Crear foros y redes sociales donde los
  estudiantes puedan interactuar entre sí.
\item
  Creación colectiva: Obtener y actuar en base a las opiniones y
  sugerencias de los usuarios para mejorar continuamente la plataforma y
  crear valor.
\end{itemize}

\subsubsection{6.4. Canales}\label{canales}

\paragraph{6.4.1. Tipo de canal}\label{tipo-de-canal}

\begin{itemize}
\tightlist
\item
  Sitio Web Oficial: El principal punto de acceso a los cursos y
  recursos de la plataforma.
\end{itemize}

\paragraph{6.4.2. Fase de canal}\label{fase-de-canal}

\begin{enumerate}
\def\labelenumi{\arabic{enumi}.}
\tightlist
\item
  Información: promoción en redes sociales, eventos y conferencias
  online.
\item
  Evaluación: Descuentos en inscripción a cursos.
\item
  Compra: A través de nuestra página web.
\item
  Entrega: A través de la personalización de cursos en la página web.
\item
  Postventa: Soporte y foros de consulta con profesores.
\end{enumerate}

\subsubsection{6.5. Segmento de clientes}\label{segmento-de-clientes}

Nuestro objetivo es crear valor para cualquier individuo que busque
mejorar sus habilidades, conocimientos y perspectivas a través de una
experiencia educativa en línea personalizada y de alta calidad. En
específico, estudiante, profesionales en busca de capacitación y
personas interesadas en aprendizaje personalizado.

\subsubsection{6.6. Estructura de costos}\label{estructura-de-costos}

\paragraph{6.6.1. Costos Fijos}\label{costos-fijos}

\textbf{Costos de Instalaciones, Medios y Equipos}

\begin{itemize}
\tightlist
\item
  Servidores de Alta Capacidad
\item
  Red de Internet de Alta Velocidad
\end{itemize}

\paragraph{6.6.2. Costos Variables}\label{costos-variables}

\textbf{Gastos de Marketing y Publicidad}

\begin{itemize}
\tightlist
\item
  Facebook Ads
\item
  Instagram Ads
\item
  Google Ads
\end{itemize}

\paragraph{6.6.3. Economías de escala}\label{economuxedas-de-escala}

A medida que aumenta el volumen de usuarios y cursos, es posible lograr
economías de escala en áreas como desarrollo de plataforma, adquisición
de contenido y marketing, lo que reduce los costos unitarios.

\subsubsection{6.7. Fuentes de ingreso}\label{fuentes-de-ingreso}

Nuestro principal flujo de ingresos proviene de las suscripciones de los
clientes, quienes acceden a nuestra plataforma para beneficiarse de
nuestros cursos en línea.

\newpage

\subsection{7. Organización: Organigrama y equipo
promotor}\label{organizaciuxf3n-organigrama-y-equipo-promotor}

\subsubsection{7.1. Organigrama}\label{organigrama}

\begin{figure}
\centering
\includegraphics[width=1\textwidth,height=\textheight]{img/Organigrama.png}
\caption{Organigrama}
\end{figure}

\subsubsection{7.2. Equipo Promotor}\label{equipo-promotor}

\textbf{Azul Noguera} \emph{Estratega Principal (CEO)}

\begin{itemize}
\tightlist
\item
  Experiencia: Más de 10 años en el sector de la educación y la
  tecnología, incluyendo roles de dirección en empresas de e-learning.
\item
  Habilidades Clave: Liderazgo estratégico, visión empresarial, gestión
  de equipos multidisciplinarios, desarrollo de negocios.
\item
  Funciones Asignadas: Toma de decisiones estratégicas, representación
  ante terceros, planificación general del negocio.
\end{itemize}

\textbf{Patricio Guledjian} \emph{Director de Tecnología (CTO)}

\begin{itemize}
\tightlist
\item
  Experiencia: Ingeniero informático con 8 años de experiencia en
  desarrollo web y aplicaciones tecnológicas para la educación.
\item
  Habilidades Clave: Desarrollo de software, gestión de infraestructura
  tecnológica, implementación de inteligencia artificial.
\item
  Funciones Asignadas: Supervisión del desarrollo de la plataforma,
  implementación de tecnologías avanzadas, garantía de calidad y
  seguridad.
\end{itemize}

\textbf{Rocio Gonzalez Cingolani} \emph{Directora de Contenido Educativo
(CCO)}

\begin{itemize}
\tightlist
\item
  Experiencia: Pedagoga con 12 años de experiencia en diseño curricular
  y desarrollo de material educativo.
\item
  Habilidades Clave: Creación de contenido didáctico, adaptación
  curricular, evaluación de aprendizaje.
\item
  Funciones Asignadas: Supervisión de la calidad del contenido,
  selección de cursos y materiales, asegurando el cumplimiento de
  estándares pedagógicos.
\end{itemize}

\newpage

\textbf{Rafael Cabre} \emph{Director de Marketing y Ventas (CMO)}

\begin{itemize}
\tightlist
\item
  Experiencia: Profesional en marketing con 7 años de experiencia en
  estrategias digitales y promoción de servicios educativos.
\item
  Habilidades Clave: Marketing digital, gestión de campañas
  publicitarias, análisis de mercado, desarrollo de marca.
\item
  Funciones Asignadas: Planificación y ejecución de estrategias de
  marketing, captación de usuarios, análisis de datos y rendimiento.
\end{itemize}

\textbf{Ian Costantini} \emph{Director Financiero (CFO)}

\begin{itemize}
\tightlist
\item
  Experiencia: Contador público con más de 15 años de experiencia en
  finanzas corporativas y gestión financiera.
\item
  Habilidades Clave: Análisis financiero, planificación presupuestaria,
  gestión de inversiones, obtención de financiamiento.
\item
  Funciones Asignadas: Elaboración de presupuestos, gestión de flujo de
  caja, análisis de rentabilidad, negociación con inversores y entidades
  financieras.
\end{itemize}

\textbf{Lucila Chaves} \emph{Jefa de Recursos Humanos (HR)}

\begin{itemize}
\tightlist
\item
  Experiencia: Especialista en recursos humanos con más de 13 años de
  experiencia en reclutamiento, capacitación y desarrollo del talento en
  empresas tecnológicas y educativas.
\item
  Habilidades Clave: Gestión de recursos humanos, desarrollo
  organizacional, legislación laboral, comunicación interna.
\item
  Funciones Asignadas: Dirección de las políticas de recursos humanos,
  reclutamiento y selección de personal, desarrollo y capacitación de
  empleados, mantenimiento de un ambiente laboral positivo y productivo.
\end{itemize}

\newpage

\subsection{8. Analisis entorno
competitivo}\label{analisis-entorno-competitivo}

\subsubsection{8.1. Empresas las cuales se dediquen a la misma actividad
o parecida a la
nuestra}\label{empresas-las-cuales-se-dediquen-a-la-misma-actividad-o-parecida-a-la-nuestra}

\begin{enumerate}
\def\labelenumi{\alph{enumi}.}
\item
  Coursera\footnote{FUENTE:
    \href{https://www.coursera.org/}{coursera.org}}
\item
  Udemy\footnote{FUENTE: \href{https://www.udemy.com/}{udemy.com}}
\item
  Duolingo\footnote{FUENTE:
    \href{https://es.duolingo.com/}{es.duolingo.com}}
\end{enumerate}

\subsubsection{8.2. Descripcion de la competencia y
producto}\label{descripcion-de-la-competencia-y-producto}

\begin{enumerate}
\def\labelenumi{\alph{enumi}.}
\item
  Coursera, fundada en octubre de 2011 por académicos de la Universidad
  de Stanford, es una destacada plataforma de educación en línea
  diseñada para ofrecer cursos masivos abiertos en línea (MOOC, por sus
  siglas en inglés). Actualmente, proporciona 3,943 cursos de pago que
  abarcan una amplia gama de disciplinas---desde idiomas y matemáticas
  hasta tecnología, salud, desarrollo personal, ciencias, artes y
  humanidades---, todos impartidos por universidades de prestigio
  mundial y acompañados de certificaciones reconocidas como el
  Mastertrack Certificate. La inscripción tiene un costo mensual de 54
  dólares estadounidenses, y la plataforma no ofrece reembolso por la
  matrícula. Sin embargo, una de las principales críticas hacia Coursera
  radica en su limitada personalización: la plataforma tiende a ofrecer
  contenido pregrabado que incluye ejercicios de opción múltiple sin
  revisión posterior, lo que puede no satisfacer a usuarios que buscan
  una experiencia de aprendizaje más interactiva y adaptada a sus
  necesidades específicas. Este aspecto representa una oportunidad
  significativa para competidores que puedan integrar sistemas de
  aprendizaje más dinámicos y personalizados.
\item
  Udemy es una prominente plataforma de e-learning establecida en 2010
  en San Francisco, California, EE. UU. Dedicada primordialmente a
  profesionales adultos, Udemy se distingue por ofrecer una extensa
  variedad de cursos en áreas como Desarrollo, Negocios, Informática y
  Software, Productividad en la Oficina, Desarrollo Personal, Diseño,
  Marketing, Estilo de Vida, Fotografía, Salud y Fitness, Música, y
  Enseñanzas Académicas.\\
  A diferencia de los tradicionales MOOC desarrollados por
  universidades, Udemy permite a los creadores independientes
  desarrollar, promocionar y monetizar sus cursos, proporcionándoles
  herramientas esenciales para gestionar y obtener ingresos a través de
  las matrículas. Aunque los cursos de Udemy no equivalen a títulos
  universitarios, son ampliamente reconocidos por mejorar habilidades
  profesionales y personales. La plataforma ofrece los cursos a un
  precio mensual de \$10.99 USD y garantiza una política de reembolso de
  30 días.\\
  Sin embargo, al igual que Coursera, Udemy carece de personalización en
  la experiencia del usuario y ofrece contenidos que no requieren una
  interacción profunda o retroalimentación detallada en las actividades,
  lo que puede limitar la profundidad del aprendizaje y la adaptación a
  las necesidades específicas del estudiante.
\item
  Duolingo, fundada en 2011 por el profesor Luis von Ahn y su estudiante
  Severin Hacker en Pittsburgh, Pensilvania, se ha consolidado como una
  plataforma líder en el aprendizaje de idiomas. Conocida por su enfoque
  gamificado y accesible, Duolingo ofrece cursos en más de 30 idiomas,
  desde ampliamente hablados como el inglés, español y francés, hasta
  idiomas menos comunes como el gaélico escocés y el hawaiano, con una
  suscripción mensual de 7.33 dólares sin posibilidad de reembolso.\\
  La plataforma se distingue por la efectividad y la conveniencia de su
  modelo educativo, que integra el aprendizaje en la rutina diaria de
  los usuarios de una manera atractiva y divertida. Con millones de
  usuarios activos a nivel mundial, Duolingo se mantiene como una de las
  aplicaciones más prominentes y estimadas en el sector del e-learning
  de idiomas.\\
  Las lecciones de Duolingo, conocidas por su brevedad, utilizan una
  variedad de ejercicios interactivos que están diseñados para mejorar
  habilidades lingüísticas como la lectura, escritura, comprensión
  auditiva y expresión oral. Cada lección incluye elementos de juego,
  tales como puntos de experiencia, niveles y vidas, que motivan a los
  usuarios a mantener un aprendizaje continuo y comprometido. Aunque
  Duolingo es ideal para quienes buscan dar sus primeros pasos en un
  nuevo idioma mediante cursos básicos presentados de forma lúdica,
  puede no cubrir necesidades avanzadas de aprendizaje lingüístico.
\end{enumerate}

\begin{longtable}[]{@{}
  >{\raggedright\arraybackslash}p{(\columnwidth - 14\tabcolsep) * \real{0.0776}}
  >{\raggedright\arraybackslash}p{(\columnwidth - 14\tabcolsep) * \real{0.0948}}
  >{\raggedright\arraybackslash}p{(\columnwidth - 14\tabcolsep) * \real{0.0862}}
  >{\raggedright\arraybackslash}p{(\columnwidth - 14\tabcolsep) * \real{0.0603}}
  >{\raggedright\arraybackslash}p{(\columnwidth - 14\tabcolsep) * \real{0.0948}}
  >{\raggedright\arraybackslash}p{(\columnwidth - 14\tabcolsep) * \real{0.2414}}
  >{\raggedright\arraybackslash}p{(\columnwidth - 14\tabcolsep) * \real{0.1466}}
  >{\raggedright\arraybackslash}p{(\columnwidth - 14\tabcolsep) * \real{0.1983}}@{}}
\toprule\noalign{}
\begin{minipage}[b]{\linewidth}\raggedright
Empresa
\end{minipage} & \begin{minipage}[b]{\linewidth}\raggedright
fundación
\end{minipage} & \begin{minipage}[b]{\linewidth}\raggedright
mercados
\end{minipage} & \begin{minipage}[b]{\linewidth}\raggedright
cursos
\end{minipage} & \begin{minipage}[b]{\linewidth}\raggedright
reembolso
\end{minipage} & \begin{minipage}[b]{\linewidth}\raggedright
contenido
\end{minipage} & \begin{minipage}[b]{\linewidth}\raggedright
certificaciones
\end{minipage} & \begin{minipage}[b]{\linewidth}\raggedright
Cliente objetivo
\end{minipage} \\
\midrule\noalign{}
\endhead
\bottomrule\noalign{}
\endlastfoot
udemy & 2010 & mundial & 67561 & si & amplia gama de disciplinas & no &
adultos profesionales \\
coursera & 2011 & mundial & 3943 & no & amplia gama de disciplinas & si
& cualquier edad \\
duolingo & 2011 & mundial & 3880 & no & idiomas & no & cualquier edad \\
\end{longtable}

\paragraph{8.2.1 Precio}\label{precio}

por dolares EEUU

\begin{longtable}[]{@{}ll@{}}
\toprule\noalign{}
Empresa & Precio mensual \\
\midrule\noalign{}
\endhead
\bottomrule\noalign{}
\endlastfoot
Coursera & \$ 54 USD \\
Udemy & \$ 15 USD \\
Duolingo & \$ 7.33 USD \\
\end{longtable}

\paragraph{8.2.2 Ingresos y beneficios.}\label{ingresos-y-beneficios.}

\subparagraph{Ingresos Operativos (facturación): por mil dolares
EEUU}\label{ingresos-operativos-facturaciuxf3n-por-mil-dolares-eeuu}

\begin{longtable}[]{@{}llll@{}}
\toprule\noalign{}
Year & Coursera & Udemy & Duolingo \\
\midrule\noalign{}
\endhead
\bottomrule\noalign{}
\endlastfoot
2023 & \$635,764 & \$728,937 & \$531,109 \\
2022 & \$523,756 & \$629,097 & \$369,495 \\
2021 & \$415,287 & \$515,657 & \$250,772 \\
2020 & \$293,511 & \$429,899 & \$161,696 \\
2019 & \$184,411 & \$276,327 & \$70,760 \\
\end{longtable}

\subparagraph{Beneficios: por mil dolares
EEUU}\label{beneficios-por-mil-dolares-eeuu}

\begin{longtable}[]{@{}llll@{}}
\toprule\noalign{}
Year & Coursera & Udemy & Duolingo \\
\midrule\noalign{}
\endhead
\bottomrule\noalign{}
\endlastfoot
2023 & \$-111,183 & \$-107,294 & \$16,067 \\
2022 & \$-170,637 & \$-153,875 & \$-59,574 \\
2021 & \$-143,089 & \$-80,026 & \$-60,135 \\
2020 & \$-65,300 & \$-77,620 & \$-15,776 \\
2019 & \$-46,001 & \$-69,703 & \$-13,554 \\
\end{longtable}

\newpage

\paragraph{8.2.3 Número de empleados.}\label{nuxfamero-de-empleados.}

\begin{longtable}[]{@{}llll@{}}
\toprule\noalign{}
Year & Cousera & Udemy & Duolingo \\
\midrule\noalign{}
\endhead
\bottomrule\noalign{}
\endlastfoot
2023 & 1,295 & 1,443 & 720 \\
2022 & 1,401 & 1,678 & 600 \\
2021 & 1,138 & 1,238 & 500 \\
2020 & 779 & 1,013 & 400 \\
2019 & n.a. & n.a & n.a \\
\end{longtable}

\newpage

\subsection{9. Definición del mercado}\label{definiciuxf3n-del-mercado}

\subsubsection{9.1. Cliente objetivo}\label{cliente-objetivo}

Nuestro servicio está diseñado para ofrecer formación complementaria a
estudiantes universitarios y profesionales que buscan especializarse en
el área tecnológica. Para definir con mayor precisión nuestro mercado
objetivo, nos centraremos en estudiantes universitarios de España que
estén interesados en cursos personalizados y adaptados a sus necesidades
de aprendizaje. Esta estrategia nos permitirá identificar y cuantificar
de manera efectiva a nuestro público objetivo, asegurando que nuestros
cursos respondan específicamente a las demandas y expectativas de este
segmento de mercado. Además, estaremos posicionados para satisfacer las
necesidades educativas actuales y futuras de los estudiantes
universitarios, ayudándoles a complementar sus estudios con habilidades
técnicas relevantes y actualizadas.

\subsubsection{9.2. Métricas}\label{muxe9tricas}

Basándonos en el estudio de mercado realizado con la encuesta mencionada
anteriormente, hemos obtenido una serie de resultados que proporcionan
información clave para abordar diferentes aspectos del negocio.

\textbf{Métrica de Referencia}

Para obtener esta métrica, evaluamos las respuestas de los usuarios en
nuestra encuesta, donde les preguntamos si recomendarían nuestro
producto. Los resultados mostraron que el 30\% de los encuestados
expresaron una disposición a recomendar un servicio similar al nuestro.

Además, es relevante destacar que el 63.3\% de los encuestados indicaron
que podrían recomendar el servicio (``Tal vez''), mientras que el 6.7\%
indicó que no estaría dispuesto a recomendarlo. Estos datos nos brindan
una visión integral de la percepción de los usuarios hacia nuestro
servicio y nos ayudan a entender las áreas donde podemos mejorar para
aumentar la tasa de recomendación.

\textbf{Métrica de Recurrencia}

La métrica de recurrencia es esencial para evaluar la fidelidad y
retención de nuestros clientes en un modelo de suscripción mensual.
Calculamos esta métrica considerando que cada cliente que adquiere
nuestra suscripción mensual la renueva durante un mínimo de un año, lo
que equivale a un total de 12 renovaciones en ese periodo.

Esta tasa de recurrencia de 12 por usuario indica que, en promedio, cada
cliente generará ingresos equivalentes a 12 meses de suscripción durante
el primer año de su relación con nuestro servicio. Este indicador es
crucial para la planificación financiera y para comprender el valor del
ciclo de vida del cliente en nuestro modelo de negocio basado en
suscripciones mensuales. Esta métrica nos permite estimar la estabilidad
y el potencial de ingresos a largo plazo derivados de nuestros clientes
recurrentes.

\subsubsection{9.3. Estudio de mercado}\label{estudio-de-mercado}

\textbf{Precio}

Nuestra plataforma se diferencia significativamente de la competencia,
como Coursera, Udemy y Duolingo, al ofrecer una experiencia altamente
personalizada y un sólido soporte que va más allá de simplemente
proporcionar contenido básico.

Lo que realmente nos distingue es nuestra atención personalizada a los
usuarios. Nuestros cursos están diseñados para complementar los estudios
universitarios, adaptándose a las necesidades únicas de cada estudiante.
Esto significa que nuestros clientes valoran la relevancia y utilidad de
nuestros contenidos en su trayectoria académica, brindándoles un mayor
valor percibido.

Además, nuestro servicio se destaca por ofrecer un sólido respaldo a
través de una interacción constante. Proporcionamos ayuda y feedback
continuo a nuestros usuarios, permitiéndoles maximizar su experiencia de
aprendizaje y obtener resultados tangibles en su formación.

Nuestro enfoque va más allá de ofrecer simplemente contenido; buscamos
brindar una experiencia completa de aprendizaje. Nuestros cursos
incluyen acceso a tutores especializados, orientación personalizada,
videos instructivos y material avanzado. Además, nuestra plataforma
utiliza la inteligencia artificial para orientar, corregir y adaptar el
contenido según las necesidades individuales de cada estudiante.

Esta combinación de recursos y tecnología garantiza que nuestros
estudiantes no solo consuman información, sino que realmente adquieran
habilidades de manera efectiva y significativa. La decisión de
establecer un precio de 60€ mensuales refleja el valor adicional y la
calidad superior que ofrecemos en comparación con otras plataformas.

\subsubsection{9.4. TAM (Mercado Total)}\label{tam-mercado-total}

Nuestro mercado total (TAM) se define por la cantidad de estudiantes
universitarios en España que podrían beneficiarse de nuestra oferta de
formación complementaria y especializaciones.

Según datos del Ministerio de Universidades de España para el curso
académico 2022-2023, podemos observar la siguiente distribución de
estudiantes: \footnote{FUENTE:
  \href{https://www.universidades.gob.es/wp-content/uploads/2023/06/Principales-resultados_EEU_2022-23.pdf}{Ministerio
  de Universidades de España}}

\begin{longtable}[]{@{}ll@{}}
\toprule\noalign{}
Nivel de estudio & Cantidad de estudiantes \\
\midrule\noalign{}
\endhead
\bottomrule\noalign{}
\endlastfoot
Grado y ciclo & 1,353,347 \\
Master & 276,518 \\
Doctorado & 92,382 \\
Total & \textbf{1,722,247} \\
\end{longtable}

Estos números representan la base potencial de clientes a los que
podemos dirigirnos con nuestro servicio educativo. Esta información
respalda nuestra estrategia de mercado y nos proporciona una visión
clara del alcance y la oportunidad dentro del sector universitario
español.

\subsubsection{9.5. SAM (Mercado Objetivo)}\label{sam-mercado-objetivo}

Para determinar nuestro SAM (Servicio Total Disponible), utilizamos los
resultados de encuestas exhaustivas que nos proporcionaron una
estimación realista del porcentaje de alumnos universitarios dispuestos
a consumir nuestro servicio. A partir de estos datos, identificamos que
aproximadamente el 42.1\% de los encuestados mostraron disposición a
pagar el precio que hemos establecido.

Basándonos en esta información, calculamos nuestro mercado objetivo
inicial tomando el total de estudiantes universitarios en España, que es
de 1,722,247, y aplicando el porcentaje de disposición a pagar (42.1\%).
Por lo tanto, nuestro SAM se estima en 1,722,247 x 42.1\% = 725,065
potenciales usuarios. Este cálculo nos proporciona una base sólida para
comprender la demanda inicial de nuestro servicio entre los estudiantes
universitarios en España que valoran la formación complementaria y
especializada que ofrecemos. Estamos enfocados en capturar una parte
significativa de este mercado objetivo inicial, aprovechando la
disposición de la mayoría de los estudiantes encuestados a invertir en
nuestra propuesta de valor.

\subsubsection{9.6. SOM (Mercado
Obtenible)}\label{som-mercado-obtenible}

El mercado obtenible de nuestra startup se define por la capacidad
técnica de nuestra infraestructura para atender de manera eficaz a
estudiantes universitarios interesados en nuestros servicios de
formación complementaria. Esta métrica se fundamenta en la capacidad de
nuestros servidores y el ancho de banda disponible.

\newpage

\textbf{Capacidad de Servidores}

Para determinar nuestro mercado obtenible, evaluamos la capacidad de
nuestros servidores para manejar el tráfico y la carga de trabajo
asociada con la entrega de nuestros cursos en línea.

Actualmente, contamos con tres servidores IBM Power S1014. Según las
especificaciones proporcionadas por IBM, cada servidor tiene las
siguientes características:

\begin{itemize}
\item
  Procesador: Hasta 8 núcleos de procesador Power10
\item
  Memoria: Hasta 1 TB de RAM
\item
  Capacidad de Red: 100 Gbps (Gigabits por segundo)
\end{itemize}

\textbf{Ancho de Banda}

El ancho de banda disponible es crucial para garantizar una experiencia
de usuario fluida y sin interrupciones durante el acceso a nuestros
cursos en línea. Según los requisitos estimados\footnote{FUENTE:
  \href{https://support.goto.com/es/joinme/help/what-are-the-bandwidth-requirements}{GoTo}},
cada estudiante necesita aproximadamente 2,5 Mbps de ancho de banda para
acceder de manera efectiva a nuestros cursos.

\textbf{Estimación del SOM}

Basándonos en las especificaciones de nuestra infraestructura técnica,
podemos estimar nuestro mercado obtenible considerando el ancho de banda
disponible en nuestros servidores.

\[
\text{Ancho de Banda Disponible} = 100 \text{ Gbps} = 100,000 \text{ Mbps}
\]

\[
\text{Ancho de Banda Requerido por Estudiante} = 2.5 \text{ Mbps}
\]

\[
\text{Número de Estudiantes por Servidor} = \frac{\text{Ancho de Banda Disponible}}{\text{Ancho de Banda Requerido por Estudiante}} = \frac{100,000}{2.5} = 40,000
\]

\[
\text{Número Total de Estudiantes (SOM)} = 40,000 \times 3 = 120,000
\]

Por lo tanto, con la capacidad de ancho de banda actual de nuestros tres
servidores (100 Gbps cada uno), podemos atender de manera efectiva hasta
120,000 estudiantes simultáneamente, considerando el requisito de 2.5
Mbps por estudiante. Esta estimación respalda nuestra capacidad para
ofrecer una experiencia educativa en línea robusta y escalable.

\newpage

\subsection{10. Marketing}\label{marketing}

\subsubsection{10.1. Medios que usaremos para vender el
producto}\label{medios-que-usaremos-para-vender-el-producto}

\textbf{Facebook}

\begin{itemize}
\tightlist
\item
  Interacción Orgánica: Nos comprometemos a mantener una presencia en
  línea sólida y dinámica a través de nuestra página oficial, la cual
  será gestionada por el Community Manager Rafael Cabré. Este
  profesional se encargará de garantizar un flujo constante de
  publicaciones diarias que se centrarán en dos áreas cruciales: la
  comunicación clara y convincente de nuestra propuesta de valor única,
  resaltando los aspectos diferenciadores de nuestro producto en el
  mercado; y la publicación de contenido relevante y actualizado
  relacionado con la educación y el estudio.
\item
  Publicidad Pautada: Implementaremos una estrategia publicitaria
  pautada en Facebook para potenciar el alcance y la visibilidad de
  nuestra plataforma. A través de campañas cuidadosamente diseñadas,
  buscaremos generar un crecimiento acelerado y dirigir un flujo
  constante de tráfico hacia nuestro sitio web.
\end{itemize}

\textbf{Instagram}

\begin{itemize}
\tightlist
\item
  Publicaciones Diarias: Trabajaremos en estrecha colaboración con
  nuestro Community Manager para desarrollar y mantener una programación
  de publicaciones diarias. Estas publicaciones se centrarán en
  transmitir nuestra propuesta de valor de manera efectiva, priorizando
  contenido visual de alta calidad, especialmente videos, para captar la
  atención de nuestra audiencia.
\item
  Campañas Dirigidas a Educadores: Implementaremos campañas
  publicitarias específicas dirigidas a educadores, destacando los
  beneficios clave de nuestra plataforma mediante comparaciones con
  métodos de enseñanza tradicionales. Enfocaremos nuestros esfuerzos en
  resaltar cómo nuestra solución puede mejorar la eficiencia, la
  interacción y los resultados en el proceso de enseñanza-aprendizaje.
\item
  Campañas Dirigidas a Alumnos: Asimismo, dirigiremos campañas
  publicitarias específicas hacia los potenciales alumnos. En estas
  campañas, pondremos énfasis en los beneficios tangibles que ofrecemos,
  como la capacidad de grabar clases, la seguridad de la plataforma y la
  disponibilidad de profesores altamente cualificados. Estos aspectos se
  presentarán de manera convincente para mostrar cómo nuestra plataforma
  puede mejorar significativamente la experiencia de aprendizaje para
  los estudiantes.
\end{itemize}

\textbf{Google Ads}

\begin{itemize}
\tightlist
\item
  Campaña de Palabras Clave: Desarrollaremos una campaña publicitaria
  enfocada en capturar la búsqueda de palabras clave relevantes para
  nuestro sector, tales como ``clases particulares'' y ``educadores
  particulares'', entre otros términos clave. Esta estrategia nos
  permitirá posicionarnos de manera destacada en los resultados de
  búsqueda y llegar a una audiencia altamente motivada en su búsqueda de
  soluciones educativas.
\item
  Constante Evaluación y Optimización: Nos comprometemos a realizar una
  evaluación continua de nuestra campaña publicitaria en Google Ads para
  garantizar una generación efectiva de tráfico hacia nuestro sitio web.
  Esto implicará un monitoreo constante de los datos de rendimiento y la
  realización de ajustes estratégicos según sea necesario para maximizar
  el impacto y la eficacia de nuestra inversión publicitaria.
\end{itemize}

\subsubsection{10.2. Costo de la
promoción}\label{costo-de-la-promociuxf3n}

En nuestro análisis de costos, emplearemos el modelo del Costo por Mil
Impresiones (CPM), una métrica fundamental en la publicidad en línea,
que cuantifica el costo que los anunciantes incurren por cada mil
impresiones de sus anuncios. Esta métrica nos permite evaluar el alcance
y la efectividad de nuestras campañas publicitarias, ofreciendo una
visión integral de nuestra inversión en marketing digital.

\textbf{Facebook Ads}

Según los datos proporcionados por Neo Attack \footnote{FUENTE:
  \href{https://neoattack.com/blog/cuanto-cuesta-facebook-ads/\#:~:text=tipo\%20de\%20empresas.-,Cu\%C3\%A1nto\%20cuesta\%20Facebook\%20Ads\%20en\%20Espa\%C3\%B1a,entre\%200.5\%20y\%203\%E2\%82\%AC.}{¿Cuánto
  cuesta Facebook Ads?}}, el CPM de una campaña en Facebook en España
generalmente oscila entre 2 y 5€.

\textbf{Instagram Ads}

De acuerdo con la información de Dinamiza Digital \footnote{FUENTE:
  \href{https://dinamizadigital.com/cuanto-cuesta-la-publicidad-en-instagram/}{Cuánto
  cuesta la publicidad en Instagram en España?}}, el CPM promedio en
España para Instagram Ads varía entre 1,97€ y 5,39€.

\textbf{Google Ads}

Aunque no se proporciona un dato específico para el CPM de Google Ads en
España en los enlaces proporcionados, se puede inferir que el CPM en
España para Google Ads podría estar en un rango similar al global,
alrededor de \$2.40.

Es importante tener en cuenta que estas cifras son estimaciones
aproximadas y que el costo real puede variar según diversos factores,
como la audiencia objetivo, el formato del anuncio y la competencia en
el mercado. Para obtener una estimación más precisa, es recomendable
considerar la colaboración con una agencia de marketing digital o el uso
de herramientas especializadas.

\subsubsection{10.3. Estimacion de personas atraidas por la
promocion}\label{estimacion-de-personas-atraidas-por-la-promocion}

Los conversion rates y los click-through rates (CTR) en las plataformas
publicitarias clave como Facebook Ads, Google Ads e Instagram Ads pueden
variar significativamente. Se observa que:

\textbf{Facebook Ads}

\textbf{Paso 1: Determinar el Número de Impresiones}

Asignamos un presupuesto de 500 euros para la campaña. Con un Costo Por
Mil impresiones (CPM) de 3 euros, el total de impresiones que podemos
obtener se calcula de la siguiente manera:

\[
\text{Impresiones Totales} = \frac{500 \, \text{euros} \times 1000}{3 \, \text{euros por mil impresiones}} = 166,666 \, \text{impresiones aproximadamente}
\]

\textbf{Paso 2: Calcular los Clics Esperados}

El CTR para Facebook Ads es del 0.73\% \footnote{FUENTE:
  \href{https://www.wordstream.com/blog/ws/2017/02/28/facebook-advertising-benchmarks}{Facebook
  Ad Benchmarks for YOUR Industry.}}, lo que indica la proporción de
usuarios que hacen clic en un anuncio después de verlo. Utilizando este
dato, estimamos el número de clics que estas impresiones generarán:

\[
\text{Clics Esperados} = 166,666 \, \text{impresiones} \times \frac{0.73}{100} = 1,217 \, \text{clics}
\]

\textbf{Paso 3: Estimar las Conversiones}

Según Wordstream, una cadena de marketing, la tasa de conversión
promedio en Facebook Ads para el sector educativo se sitúa en torno al
13.58\% \footnote{FUENTE:
  \href{https://www.wordstream.com/blog/ws/2017/02/28/facebook-advertising-benchmarks}{Facebook
  Ad Benchmarks for YOUR Industry.}}. Utilizando este dato, calculamos
el número de conversiones esperadas de los usuarios que hicieron clic:

\[
\text{Conversiones Esperadas} = 1,217 \, \text{clics} \times \frac{13.58}{100} = 165 \, \text{conversiones}
\]

Este análisis proporciona una visión clara del impacto potencial de una
inversión de 500 euros en una campaña de Facebook Ads dirigida al sector
educativo.

\begin{figure}
\centering
\includegraphics[width=0.8\textwidth,height=\textheight]{img/facebook.png}
\caption{La tasa de conversion de todas las industrias de Facebook Ads.}
\end{figure}

\textbf{Google Ads}

\textbf{Paso 1: Determinar el Número de Impresiones}

Asignamos un presupuesto de 500 euros para la campaña. Con un Costo Por
Mil impresiones (CPM) de 2.40 euros, el total de impresiones que podemos
obtener se calcula de la siguiente manera:

\[
\text{Impresiones Totales} = \frac{500 \, \text{euros} \times 1000}{2.40 \, \text{euros por mil impresiones}} = 208,333 \, \text{impresiones aproximadamente}
\]

\textbf{Paso 2: Calcular los Clics Esperados}

El CTR para Google Ads es del 6.17\% \footnote{FUENTE:
  \href{https://www.wordstream.com/blog/ws/2022/05/18/search-advertising-benchmarks}{2022
  Google Ads \& Microsoft Ads Benchmarks for Every Industry.}}, lo que
indica la proporción de usuarios que hacen clic en un anuncio después de
verlo. Utilizando este dato, estimamos el número de clics que estas
impresiones generarán:

\[
\text{Clics Esperados} = 208,333 \, \text{impresiones} \times \frac{6.17}{100} = 12,843 \, \text{clics}
\]

\textbf{Paso 3: Estimar las Conversiones}

En el caso de Google Ads, la tasa de conversión promedio para el sector
educativo se sitúa alrededor del 5.93\% para la búsqueda (Search)
\footnote{FUENTE:
  \href{https://www.wordstream.com/blog/ws/2022/05/18/search-advertising-benchmarks}{2022
  Google Ads \& Microsoft Ads Benchmarks for Every Industry.}}.
Utilizando este dato, calculamos el número de conversiones esperadas de
los usuarios que hicieron clic:

\[
\text{Conversiones Esperadas} = 12,843 \, \text{clics} \times \frac{5.93}{100} = 762 \, \text{conversiones}
\]

Este análisis proporciona una visión clara del impacto potencial de una
inversión de 500 euros en una campaña de Google Ads dirigida al sector
educativo.

\newpage

\textbf{Instagram Ads}

\textbf{Paso 1: Determinar el Número de Impresiones}

Asignamos un presupuesto de 500 euros para la campaña. Con un Costo Por
Mil impresiones (CPM) de 3.68 euros, el total de impresiones que podemos
obtener se calcula de la siguiente manera:

\[
\text{Impresiones Totales} = \frac{500 \, \text{euros} \times 1000}{3.68 \, \text{euros por mil impresiones}} = 135,870 \, \text{impresiones aproximadamente}
\]

\textbf{Paso 2: Calcular los Clics Esperados}

Respecto al CTR para Instagram Ads, este se encuentra en el 0.52\%
\footnote{FUENTE:
  \href{https://lorenzo-gonzalez.com/instagram-analytics-las-9-metricas-mas-importantes-para-medir-tu-exito/\#:~:text=Tasas\%20de\%20clics\%20de\%20anuncios\%20(CTR),-El\%20objetivo\%20principal&text=Las\%20tasas\%20de\%20clics\%20promedio,\%2C52\%25\%2Cseg\%C3\%BAn\%20Adstage.}{Instagram
  Analytics}}. Haciendo uso de este dato, estimamos el número de clics
que estas impresiones generarán:

\[
\text{Clics Esperados} = 135,870 \, \text{impresiones} \times \frac{0.52}{100} = 706 \, \text{clics}
\]

\textbf{Paso 3: Estimar las Conversiones}

Aunque los datos específicos para el sector educativo no están
disponibles, es importante tener en cuenta que el conversion rate
promedio para Instagram Ads se sitúa en aproximadamente el 1.08\%
\footnote{FUENTE:
  \href{https://visme.co/blog/es/instagram-ads/}{Instagram Ads: ¿Cómo
  crearlos en 2023 para convertir?}}. Utilizando este dato, calculamos
el número de conversiones esperadas de los usuarios que hicieron clic:

\[
\text{Conversiones Esperadas} = 706 \, \text{clics} \times \frac{1.08}{100} = 8 \, \text{conversiones}
\]

Este análisis proporciona una visión clara del impacto potencial de una
inversión de 500 euros en una campaña de Instagram Ads.

\textbf{Numero final de clientes}

Es crucial tener en cuenta que las tasas de conversión y los CTR pueden
variar según diversos factores, como la estrategia de marketing, la
calidad del contenido y la relevancia de las campañas publicitarias en
cada plataforma. Por lo tanto, es esencial realizar un seguimiento
continuo y ajustar nuestras estrategias según los resultados obtenidos
para maximizar el rendimiento de nuestras inversiones en publicidad
digital.

En resumen, al considerar las estimaciones de clientes generados a
través de las campañas publicitarias en Facebook Ads, Instagram Ads y
Google Ads, se proyecta un total de 935 nuevos clientes.

\newpage

\subsection{11. Produccion y
operaciones}\label{produccion-y-operaciones}

\subsubsection{11.1. Servicio Minimo Viable y
localizacion}\label{servicio-minimo-viable-y-localizacion}

\textbf{Desarrollo de Cursos Originales}

Crearemos cursos de tecnología completamente originales, diseñados
internamente por nuestro equipo experto. Estos cursos se adaptarán a
diferentes estilos de aprendizaje e intereses de los usuarios,
permitiendo una mayor personalización y relevancia.

\textbf{Aprendizaje Adaptativo Básico}

Implementaremos funcionalidades de aprendizaje adaptativo que ajusten el
nivel de dificultad y el ritmo del curso según el progreso del
estudiante. Este proceso inicialmente será supervisado por instructores
y evolucionará hacia un enfoque más automatizado con algoritmos simples.

\textbf{Encuestas de Retroalimentación}

Realizaremos encuestas periódicas para recopilar información detallada
sobre la experiencia de aprendizaje de los estudiantes. Esto nos
permitirá identificar áreas de mejora y adaptar de manera personalizada
el contenido y la estructura de nuestros cursos.

\textbf{Comunidad}

Facilitaremos un espacio interactivo donde los estudiantes puedan
conectar entre sí y con los instructores. Este entorno permitirá
discusiones, intercambio de recursos y brindará una plataforma para
recibir retroalimentación directa sobre la experiencia de aprendizaje.

Con este enfoque, lanzaremos un producto inicial sólido que destaque por
la calidad y la personalización de nuestros cursos originales. A medida
que obtengamos retroalimentación y validemos nuestro enfoque, podremos
expandirnos a otras áreas de aprendizaje y mejorar la funcionalidad de
la plataforma.

\textbf{Localización}

En los primeros años de funcionamiento, contaremos con un piso alquilado
que servirá como lugar para alojar nuestros servidores y como oficina de
trabajo. El alquiler mensual de este espacio está establecido en 7700€
\footnote{FUENTE:
  \href{https://www.idealista.com/inmueble/104686841/}{idealista.com}}.
Además, para acondicionar adecuadamente este espacio y hacerlo funcional
como área de trabajo y alojamiento para los servidores, planeamos
invertir aproximadamente 3000€ en mobiliario y equipamiento de oficina.
Esta inversión incluirá la compra de escritorios, sillas, estanterías y
otros elementos necesarios para un entorno de trabajo cómodo y
eficiente.

Es fundamental destacar que el costo de amueblar el espacio se
amortizará a lo largo de los próximos cinco años, reflejando la vida
útil esperada de los muebles y equipos de oficina. Esta inversión
inicial nos permitirá contar con un entorno adecuado para operar
eficazmente y albergar nuestra infraestructura tecnológica.

\subsubsection{11.2. Instalaciones, Medios y
Equipos}\label{instalaciones-medios-y-equipos}

\textbf{Medios Técnicos:}

\begin{itemize}
\tightlist
\item
  Servidores de Alta Capacidad: Con el fin de asegurar la escalabilidad
  y la robustez de nuestra plataforma de educación en línea, planeamos
  adquirir tres servidores IBM Power S1014 \footnote{FUENTE:
    \href{https://www.ibm.com/es-es/products/power-s1014}{ibm.com}}.
  Estos servidores son reconocidos por su alto rendimiento en
  procesamiento y su capacidad de expansión, lo que los hace ideales
  para manejar grandes volúmenes de datos y transacciones en tiempo
  real. El costo estimado de cada servidor es de aproximadamente
  15,000€, lo que resulta en una inversión inicial total de 45,000€ para
  esta adquisición.
\end{itemize}

Equipos Informáticos: Para respaldar las operaciones de nuestro equipo
de desarrollo y soporte técnico, hemos invertido en computadoras de
última generación con capacidades de procesamiento y memoria adecuadas
para tareas exigentes en el desarrollo de software y el mantenimiento de
la plataforma. Estimamos un costo de aproximadamente 2,000€ por
computadora, lo que totaliza una inversión inicial de 12,000€ para seis
equipos. Dado que estos equipos tienen una vida útil de aproximadamente
5 años, planeamos amortizar esta inversión a lo largo de este periodo,
distribuyendo los costos a lo largo del tiempo para reflejar
adecuadamente su valor a lo largo de su ciclo de vida útil.

\textbf{Infraestructura:}

\begin{itemize}
\tightlist
\item
  Red de Internet de Alta Velocidad: Esencial para el acceso sin
  interrupciones y el funcionamiento eficaz de la plataforma, contamos
  con un servicio de internet de fibra óptica que ofrece velocidades
  óptimas para el desarrollo y la entrega de contenido en línea. Esto es
  crucial para el trabajo remoto del equipo y la experiencia del usuario
  final.
\end{itemize}

\textbf{Tecnología:}

\begin{itemize}
\tightlist
\item
  Inteligencia Artificial (IA): Bajo la dirección de nuestro
  especialista en tecnología, desarrollaremos algoritmos avanzados de IA
  que permitirán personalizar la experiencia de aprendizaje de los
  estudiantes. Estos algoritmos estarán diseñados para analizar el
  progreso del usuario y adaptar el contenido educativo de manera
  proactiva, mejorando así la retención del conocimiento y la
  satisfacción del estudiante. Esta inversión no solo refuerza nuestro
  compromiso con la educación personalizada sino que también coloca a
  nuestra plataforma a la vanguardia de la innovación tecnológica en
  educación.
\end{itemize}

\textbf{Inversión:}

\begin{itemize}
\tightlist
\item
  Costos de Implementación: Además de la compra de los servidores, se
  invertirán aproximadamente 2,000€ en la instalación y configuración
  inicial, asegurando que la infraestructura tecnológica esté
  óptimamente preparada para soportar nuestras operaciones desde el
  inicio.
\end{itemize}

\begin{longtable}[]{@{}
  >{\raggedright\arraybackslash}p{(\columnwidth - 4\tabcolsep) * \real{0.4105}}
  >{\raggedright\arraybackslash}p{(\columnwidth - 4\tabcolsep) * \real{0.4211}}
  >{\raggedright\arraybackslash}p{(\columnwidth - 4\tabcolsep) * \real{0.1684}}@{}}
\toprule\noalign{}
\begin{minipage}[b]{\linewidth}\raggedright
Categoría
\end{minipage} & \begin{minipage}[b]{\linewidth}\raggedright
Detalle
\end{minipage} & \begin{minipage}[b]{\linewidth}\raggedright
Costo (euros)
\end{minipage} \\
\midrule\noalign{}
\endhead
\bottomrule\noalign{}
\endlastfoot
Servidores de Alta Capacidad & 3 x IBM Power S1014 & 45,000 \\
Instalación y Configuración & Configuración inicial de servidores &
2,000 \\
Equipos informáticos & 6 x Dell Precision 5470 & 12,000 \\
Equipamiento oficinas & Sillas, escritorios, estanterías & 3,000 \\
\textbf{Total Estimado} & & \textbf{62,000} \\
\end{longtable}

Esta planificación estratégica de la inversión en infraestructura
tecnológica crea una base sólida para el lanzamiento y la expansión de
nuestra plataforma educativa en línea. Al invertir en servidores de alta
capacidad y desarrollar internamente nuestras soluciones de IA, estamos
asegurando que nuestra plataforma no solo cumpla con los requisitos
actuales de los usuarios sino que también esté preparada para adaptarse
a las necesidades futuras, proporcionando así una experiencia educativa
en línea superior y altamente confiable.

\subsubsection{11.3. Sistema de
operaciones}\label{sistema-de-operaciones}

\textbf{Planificación Estratégica}

\begin{itemize}
\tightlist
\item
  Definir objetivos claros para el MVP, incluyendo las funcionalidades
  esenciales que se deben implementar.
\item
  Establecer un cronograma detallado con hitos específicos para el
  desarrollo y lanzamiento.
\item
  Desarrollo e Implementación de la Plataforma Web
\end{itemize}

\textbf{Desarrollo de Contenido de Cursos}

\begin{itemize}
\tightlist
\item
  Diseño y desarrollo de módulos educativos interactivos por parte de
  nuestros expertos en creación de contenido.
\item
  Integración de multimedia y recursos interactivos para mejorar la
  experiencia de aprendizaje.
\item
  Revisión y validación del contenido por expertos en la materia para
  garantizar su calidad y relevancia.
\end{itemize}

\textbf{Aprobación y Publicación}

\begin{itemize}
\tightlist
\item
  Realizar pruebas exhaustivas de la plataforma y los contenidos para
  asegurar su funcionalidad y precisión.
\item
  Obtener las aprobaciones necesarias de las partes interesadas para
  asegurar que el producto cumple con las expectativas y requisitos
  reglamentarios.
\item
  Publicar el contenido en la plataforma, asegurando que esté accesible
  y optimizado para diferentes dispositivos.
\end{itemize}

\textbf{Lanzamiento de los Cursos}

\begin{itemize}
\tightlist
\item
  Planificar una estrategia de lanzamiento que incluya eventos
  promocionales y colaboraciones con influencers o expertos del sector.
\item
  Monitorear la respuesta de los usuarios y ajustar rápidamente
  cualquier aspecto del servicio según sea necesario.
\end{itemize}

\textbf{Marketing y Publicidad}

\begin{itemize}
\tightlist
\item
  Desarrollar y ejecutar una campaña de marketing digital para crear
  conciencia y atraer a los usuarios objetivo.
\item
  Utilizar estrategias de marketing de contenidos, publicidad pagada y
  redes sociales para maximizar el alcance y la captación.
\item
  Establecer métricas de seguimiento para evaluar el impacto de las
  actividades de marketing y ajustar las tácticas en función de los
  resultados obtenidos.
\end{itemize}

\newpage

\subsection{12. Financiación y análisis
económico-financiero}\label{financiaciuxf3n-y-anuxe1lisis-econuxf3mico-financiero}

\subsubsection{12.1. Necesidades económicas del
proyecto}\label{necesidades-econuxf3micas-del-proyecto}

\begin{longtable}[]{@{}lr@{}}
\toprule\noalign{}
Categoria & Costo (euros) \\
\midrule\noalign{}
\endhead
\bottomrule\noalign{}
\endlastfoot
\textbf{Instalaciones, Medios y Equipos} & \\
Servidores & 45,000 \\
Instalación y Configuración & 2,000 \\
Equipos informáticos & 12,000 \\
Equipamiento oficinas & 3,000 \\
Renta (Anual) & 92,400 \\
\textbf{Total estimado} & \textbf{154,400} \\
& \\
\textbf{Costos de marketing} & \\
Facebook ads & 500 \\
Instagram ads & 500 \\
Google ads & 500 \\
\textbf{Total Estimado} & \textbf{1,500} \\
& \\
\textbf{Total De Inversion} & \textbf{155,900} \\
\end{longtable}

\subsubsection{12.2 Fuentes de
Financiacion}\label{fuentes-de-financiacion}

\textbf{Crowdfunding de Capital}

La empresa buscará obtener fondos mediante una campaña de crowdfunding,
involucrando a pequeños inversores que contribuyan financieramente a
cambio de acciones de la empresa. Estas acciones representarán su
participación en la plataforma. Los términos y condiciones asociados
incluirán aspectos como el porcentaje de participación en la empresa,
derechos de voto en asuntos importantes, acceso a dividendos en caso de
rentabilidad y posibles derechos preferenciales en caso de liquidación.
La empresa ha asegurado un capital de 80,000€ mediante crowdfunding, lo
que equivale al 50\% del capital social.

\textbf{Inversión de Capital del Equipo Promotor}

El equipo fundador de la plataforma también contribuirá con una parte
significativa del capital necesario para financiar el proyecto. Esta
inversión refleja un sólido compromiso por parte del equipo con la
visión y el éxito de la plataforma. La inversión personal del equipo
fundador resalta la confianza en el potencial de la plataforma y puede
servir como una señal positiva para los inversores externos interesados
en participar en nuestra empresa. El equipo fundador aportará 80,000€ de
capital personal, lo que equivale al 50\% del capital social de la
empresa.

La inversión total del capital social de la empresa después de estas
contribuciones ascenderá a 160,000€, superando así nuestra inversión
inicial en el proyecto. Este aumento en el capital social se destina a
proporcionar márgenes de seguridad durante los primeros años de
operación de la empresa.

\newpage

\subsubsection{12.3. Presupuesto (en euros)}\label{presupuesto-en-euros}

\begin{longtable}[]{@{}
  >{\raggedright\arraybackslash}p{(\columnwidth - 12\tabcolsep) * \real{0.2340}}
  >{\raggedright\arraybackslash}p{(\columnwidth - 12\tabcolsep) * \real{0.0957}}
  >{\raggedright\arraybackslash}p{(\columnwidth - 12\tabcolsep) * \real{0.1170}}
  >{\raggedright\arraybackslash}p{(\columnwidth - 12\tabcolsep) * \real{0.1383}}
  >{\raggedright\arraybackslash}p{(\columnwidth - 12\tabcolsep) * \real{0.1383}}
  >{\raggedright\arraybackslash}p{(\columnwidth - 12\tabcolsep) * \real{0.1383}}
  >{\raggedright\arraybackslash}p{(\columnwidth - 12\tabcolsep) * \real{0.1383}}@{}}
\toprule\noalign{}
\begin{minipage}[b]{\linewidth}\raggedright
Años
\end{minipage} & \begin{minipage}[b]{\linewidth}\raggedright
0
\end{minipage} & \begin{minipage}[b]{\linewidth}\raggedright
1
\end{minipage} & \begin{minipage}[b]{\linewidth}\raggedright
2
\end{minipage} & \begin{minipage}[b]{\linewidth}\raggedright
3
\end{minipage} & \begin{minipage}[b]{\linewidth}\raggedright
4
\end{minipage} & \begin{minipage}[b]{\linewidth}\raggedright
5
\end{minipage} \\
\midrule\noalign{}
\endhead
\bottomrule\noalign{}
\endlastfoot
Ingreso de ventas & & 673,200 & 1,548,360 & 2,686,020 & 4,165,020 &
6,087,720 \\
Gastos totales & & 155,900 & 155,900 & 155,900 & 155,900 & 155,900 \\
EBITDA & & 517,300 & 1,392,460 & 2,530,120 & 4,009,120 & 5,931,820 \\
Gastos financieros & & 0 & 0 & 0 & 0 & 0 \\
Amortizaciones & & 31,180 & 31,180 & 31,180 & 31,180 & 31,180 \\
BAIT & & 486,120 & 1,361,280 & 2,498,940 & 3,977,940 & 5,900,640 \\
Impuesto 15\% & & 72,918 & 204,192 & 374,841 & 596,691 & 885,096 \\
Inversion & 155,900 & 0 & 0 & 0 & 0 & \\
\textbf{Beneficio Neto} & & \textbf{413,202} & \textbf{1,157,088} &
\textbf{2,124,099} & \textbf{3,381,249} & \textbf{5,015,544} \\
\end{longtable}

\begin{longtable}[]{@{}lll@{}}
\toprule\noalign{}
Número de Clientes & Métrica de Recurrencia & Conversión Anual \\
\midrule\noalign{}
\endhead
\bottomrule\noalign{}
\endlastfoot
935 & 12 & 11,220 \\
\end{longtable}

\begin{longtable}[]{@{}
  >{\raggedright\arraybackslash}p{(\columnwidth - 10\tabcolsep) * \real{0.3827}}
  >{\raggedright\arraybackslash}p{(\columnwidth - 10\tabcolsep) * \real{0.1235}}
  >{\raggedright\arraybackslash}p{(\columnwidth - 10\tabcolsep) * \real{0.1235}}
  >{\raggedright\arraybackslash}p{(\columnwidth - 10\tabcolsep) * \real{0.1235}}
  >{\raggedright\arraybackslash}p{(\columnwidth - 10\tabcolsep) * \real{0.1235}}
  >{\raggedright\arraybackslash}p{(\columnwidth - 10\tabcolsep) * \real{0.1235}}@{}}
\toprule\noalign{}
\begin{minipage}[b]{\linewidth}\raggedright
Años
\end{minipage} & \begin{minipage}[b]{\linewidth}\raggedright
1
\end{minipage} & \begin{minipage}[b]{\linewidth}\raggedright
2
\end{minipage} & \begin{minipage}[b]{\linewidth}\raggedright
3
\end{minipage} & \begin{minipage}[b]{\linewidth}\raggedright
4
\end{minipage} & \begin{minipage}[b]{\linewidth}\raggedright
5
\end{minipage} \\
\midrule\noalign{}
\endhead
\bottomrule\noalign{}
\endlastfoot
\textbf{Conversión por Publicidad} & 11,220 & 11,220 & 11,220 & 11,220 &
11,220 \\
100\% Retenciones & 0 & 11,220 & 25,806 & 44,768 & 69,418 \\
30\% Referencia & 0 & 3,366 & 7,742 & 13,430 & 20,826 \\
\textbf{Conversión Total Anual} & 11,220 & 25,806 & 44,768 & 69,418 &
101,464 \\
\textbf{Ingresos de Ventas} & 673,200 & 1,548,360 & 2,686,068 &
4,165,100 & 6,087,810 \\
\end{longtable}

\end{document}
